\section{Results}
For the final results the actions described in Section \ref{subsection:usingWeka} were executed.

\subsection{Predictors of adherence: First session}
The ranking of the attributes gave the ranking as seen in Table \ref{table:adherenceFirstAttributes}. As already expected the code 91 is a valuable attribute, this attribute is also used for the actual adherence value. Code 90 has to be used before you can use code 91 because of the layout of the website, so it is good to see that this code is just below. Then the action count and session length are the best attributes, which are also expected to have influence. The attributes that are not mentioned in the table have a score of 0, which means they give (almost) no information.

\begin{table}[]
	\centering
	\caption{First session adherence \\attributes ranking}
	\label{table:adherenceFirstAttributes}
	\begin{tabular}{@{}ll@{}}
		\toprule
		\textbf{Score} & \textbf{Attribute} \\ \midrule
		0.3535         & code91             \\
		0.102          & code90             \\
		0.0465         & actionCount        \\
		0.0395         & sessionLength      \\
		0.257          & code10             \\
		0.0176         & code52             \\
		0.0105         & code53             \\ \bottomrule
	\end{tabular}
\end{table}

Next the \code{ADTree} classifier has been ran, the decision tree can be found in Listing \ref{listing:adherenceFirstTree} and the confusion matrix can be found in Table \ref{table:adherenceFirstMatrix}. Once again code 91 is important, but now also the actions per hour is used often in the tree. This is probably because the actions per hour combines the action count and session length attributes, so therefore it might be more valuable than deciding on only one of them. 

\begin{lstlisting}[caption={Adherence first session, ADTree decision tree}, label=listing:adherenceFirstTree, float=htpb, language=xml]
: 0.03
|	(1)code91 < 0.5: 0.399
|	|	(2)actionsPerHour < 177.5: -0.088
|	|	|	(3)code51 < 0.5: 0.035
|	|	|	|	(8)actionsPerHour < 35.5: -0.475
|	|	|	|	(8)actionsPerHour >= 35.5: 0.061
|	|	|	|	|	(9)code10 < 4.5: -0.056
|	|	|	|	|	(9)code10 >= 4.5: 0.789
|	|	|	(3)code51 >= 0.5: -1.042
|	|	|	(6)actionCount < 14.5: 0.044
|	|	|	(6)actionCount >= 14.5: -0.438
|	|	|	|	(7)actionsPerHour < 54.5: 0.397
|	|	|	|	(7)actionsPerHour >= 54.5: -0.652
|	|	(2)actionsPerHour >= 177.5: 0.593
|	|	|	(4)sessionLength < 0.5: -0.669
|	|	|	(4)sessionLength >= 0.5: 0.324
|	|	|	|	(5)code30 < 0.5: 1.042
|	|	|	|	(5)code30 >= 0.5: -0.663
|	(1)code91 >= 0.5: -3.364
\end{lstlisting}

The confusion matrix shows that 88 instances have been correctly classified as adherent and 136 instances have been correctly classified as not adherent. 58 instances have been classified as not adherent, but should be adherent (19.27\%), and 19 instances have been classified as adherent while they should not be (6.31\%). This brings the performance of the classification to 74.42\%.

\begin{table}[]
	\centering
	\caption{Adherence first session, \\confusion matrix}
	\label{table:adherenceFirstMatrix}
	\begin{tabular}{@{}ll|l@{}}
		\toprule
		\textbf{A} & \textbf{B} & \textless-- classified as \\ \midrule
		88         & 58         & a = Adherent              \\
		19         & 136        & b = NotAdherent           \\ \bottomrule
	\end{tabular}
\end{table}


\subsection{Predictors of adherence: Days between sessions}
For predicting the adherence based on the number of days between the first three sessions the attribute ranking and decision tree has been produced again. The ranking shows that both attributes do not have much information gain, so this already tells that the performance is probably worse. The classification with \code{BFTree} produces the tree as shown in Listing \ref{listing:adherenceSessions}. The tree only uses the number of days between the first and second session.

\begin{lstlisting}[caption={Adherence days between sessions, BFTree decision tree}, label=listing:adherenceSessions, float=htpb, language=xml]
loginTimeFirstSecond < 201.5: Adherent(133.82/127.27)
loginTimeFirstSecond >= 201.5: NotAdherent(27.73/12.18)
\end{lstlisting}

The confusion matrix shows that 100 instance have been correctly classified as adherent and 114 instance have been correctly classified as not adherent. 46 instances have been classified as not adherent, but should be adherent (15.28\%), and 41 instances have been classified as adherent while they should not be (13.92\%). This classification has a performance of 71.10\%.

\begin{table}[]
	\centering
	\caption{Adherence days between sessions, \\confusion matrix}
	\label{table:adherenceSessionsMatrix}
	\begin{tabular}{@{}ll|l@{}}
		\toprule
		\textbf{A} & \textbf{B} & \textless-- classified as \\ \midrule
		100        & 46         & a = Adherent              \\
		41         & 114        & b = NotAdherent           \\ \bottomrule
	\end{tabular}
\end{table}


\subsection{Prediction of days between sessions}
First the results for the prediction of the days between the first and second session. Listing \ref{listing:sessionsFirstSecond} shows the formula generated by the \code{LinearRegression} algorithm and Table \ref{table:sessionsFirstSecond} shows the performance of the regression formula. The generated formula has a correlation of 0.0499, which is very close to zero, this means that the prediction of the number of days is quite bad.

\begin{lstlisting}[caption={Formula prediction of days between session one and two}, label=listing:sessionsFirstSecond, float=htpb, language=xml]
loginTimeFirstSecond =
	-7.0851 * code10 +
	24.6264 * code21 +
	597.643  * code30 +
	-620.1051 * code31 +
	754.1389 * code34 +
	-106.3367 * code35 +
	46.7325 * code50 +
	-93.1106 * code51 +
	-24.8427 * code90 +
	96.2365
\end{lstlisting}

\begin{table}[]
	\centering
	\caption{Performance prediction of days between session one and two}
	\label{table:sessionsFirstSecond}
	\begin{tabular}{@{}ll@{}}
		\toprule
		\textbf{Performance indicator}  & \textbf{Value} \\ \midrule
		Correlation coefficient         & 0.0499         \\
		Mean absolute error             & 90.4414        \\
		Root mean squared error         & 128.9717       \\
		Relative absolute error         & 106.3579\%     \\
		Root relative squared error     & 109.8663\%     \\
		Total number of instances       & 181            \\
		Ignored class unknown instances & 120            \\ \bottomrule
	\end{tabular}
\end{table}

Now the results of the prediction of the days between the second and third sessions. Listing \ref{listing:sessionsSecondThird} shows the formula generated by the \code{LinearRegression} algorithm and Table \ref{table:sessionsSecondThird} shows the performance of the regression formula. The generated formula has a correlation of -0.1348, which is very close to zero, this means that the prediction of the number of days is quite bad. Note that the correlation is below zero for this regression, and for the other one it was above zero.

\begin{lstlisting}[caption={Formula prediction of days between session one and two}, label=listing:sessionsSecondThird, float=htpb, language=xml]
loginTimeSecondThird =
	-46.866  * code30 +
	6.3303 * actionCount +
	-2.9964 * sessionLength +
	58.1361
\end{lstlisting}

\begin{table}[]
	\centering
	\caption{Performance prediction of days between session one and two}
	\label{table:sessionsSecondThird}
	\begin{tabular}{@{}ll@{}}
		\toprule
		\textbf{Performance indicator}  & \textbf{Value} \\ \midrule
		Correlation coefficient         & -0.1348        \\
		Mean absolute error             & 92.7638        \\
		Root mean squared error         & 124.7705       \\
		Relative absolute error         & 106.4739\%     \\
		Root relative squared error     & 108.7412\%     \\
		Total number of instances       & 128            \\
		Ignored class unknown instances & 173            \\ \bottomrule
	\end{tabular}
\end{table}






