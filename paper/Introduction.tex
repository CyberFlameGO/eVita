\section{Introduction}
Online health technology, also called eHealth, is becoming more common. Because declining birth rates and a longer life expectancy, the number of elderly people to provide healtcare for is increasing. A platform that is trying to provide online healthcare is e-Vita\footnote{https://www.e-vita.nl/}. This website helps people to deal with chronic deceases by providing information, tracking measurements and settings goals. The website tracks the users by logging the taken actions. In this research this data is used to predict the adherence of users, this means that we will be predicting if the user keeps using the website from the data of the first session.

The used dataset is of people with Diabetes and is provided to us by the University of Twente. The dataset contains rows, that each have a userid, code, date and optionally extra information. The code describes which action the user has taken, for example visiting the homepage or adding a personal goal. Main questions that will be researched are "What are the main predictors for adherence?" and "What are the predictors for the number of days between the first and second session?". To answer these questions we first need to have a definition of adherence, which will be described in Section \ref{section:materialsAndMethods}. Based on the definition the adherence of the user will be determined, which will be used as the actual adherence value. To predict the adherence a combination of features derived from the first session will be used. A session is defined as all actions of a user, that are within 30 minutes of each other. Features can be for example if a certain function of e-Vita is used, the number of actions in the first session and the length of the session.

Next to this the predictors for the number of days between the first and second session are researched, this is based on the same features and uses regression to test for influence of the features. In this paper the process of transforming the data to features and classes is described and the use of the Weka \cite{weka} tool to test classification of the users.