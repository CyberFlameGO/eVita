\section{Related Work}
This is, of course, not the first study researching adherence of the users of a certain application or website. Carter et al. performed a pilot study with the aim to collect acceptability and feasibility outcomes of a self-monitoring weight management intervention delivered by a smartphone app, compared to a website and paper diary \cite{carter2013adherence}. 

A sample of 128 overweight volunteers was randomized to receive a weight management intervention delivered by the smartphone app, website, or paper diary. The smartphone app intervention, My Meal Mate (MMM), was developed by the research team using an evidence-based behavioral approach. The app incorporates goal setting, self-monitoring of diet and activity, and feedback via weekly text message. The website group used an existing commercially available slimming website from a company called Weight Loss Resources who also provided the paper diaries. The comparator groups delivered a similar self-monitoring intervention to the app, but by different modes of delivery. Participants were recruited by email, intranet, newsletters, and posters from large local employers. Trial duration was 6 months. The intervention and comparator groups were self-directed with no ongoing human input from the research team. The only face-to-face components were at baseline enrollment and brief follow-up sessions at 6 weeks and 6 months to take anthropometric measures and administer questionnaires.

Trial retention was 40/43 (93\%) in the smartphone group, 19/42 (55\%) in the website group, and 20/43 (53\%) in the diary group at 6 months. Adherence was statistically significantly higher in the smartphone group with a mean of 92 days (SD 67) of dietary recording compared with 35 days (SD 44) in the website group and 29 days (SD 39) in the diary group (P<.001). Self-monitoring declined over time in all groups. In an intention-to-treat analysis using baseline observation carried forward for missing data, mean weight change at 6 months was -4.6 kg (95% CI –6.2 to –3.0) in the smartphone app group, –2.9 kg (95% CI –4.7 to –1.1) in the diary group, and –1.3 kg (95% CI –2.7 to 0.1) in the website group. BMI change at 6 months was –1.6 kg/m2 (95% CI –2.2 to –1.1) in the smartphone group, –1.0 kg/m2 (95% CI –1.6 to –0.4) in the diary group, and –0.5 kg/m2 (95% CI –0.9 to 0.0) in the website group. Change in body fat was –1.3% (95% CI –1.7 to –0.8) in the smartphone group, –0.9% (95% CI –1.5 to –0.4) in the diary group, and –0.5% (95% CI –0.9 to 0.0) in the website group. Their results show that the MMM app was an acceptable and feasible weight loss intervention and a full RCT of this approach is warranted.

Another study conducted in 2004 describes participant engagement and retention with a stage-based physical activity website in a workplace setting \cite{leslie2005engagement}. This research analyzed data from participants in the website condition of a randomized trial designed to test the efficacy of a print- vs. website-delivered intervention. They received four stage-targeted e-mails over 8 weeks, with hyperlinks to the website. Both objective and self-reported website use data were collected and analyzed.

Overall, 327 were randomized to the website condition and 250 (76\%) completed the follow-up survey. Forty-six percent (n = 152) visited the website over the trial period. A total of 4,114 hits to the website were recorded. Participants who entered the site spent on average 9 minutes per visit and viewed 18 pages. Website use declined over time; 77\% of all visits followed the first e-mail.

These results conclude that limited website engagement, despite the perceived usefulness of the materials, demonstrates possible constraints on the use of e-mails and websites in delivering health behavior change programs. In the often-cluttered information environment of workplaces, issues of engagement and retention in website-delivered programs require attention.