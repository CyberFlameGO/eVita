\section{Introduction}
Online health technology, also called eHealth, is becoming more present and accepted as a new medium for healthcare. Due to declining birth rates and a longer life expectancy, the number of elderly people needing healthcare is increasing while there is a shortage of people who are able to provide this. A platform that is trying to provide online healthcare is e-Vita\footnote{https://www.e-vita.nl/}. This website helps people to deal with chronic diseases by providing information, tracking measurements and settings goals. The website tracks the users by logging the actions performed. In this research, this data is used to predict the adherence of users. We are interested in predicting if the user keeps using the website by investigating data of the first session for this user.

The dataset used for this research is of people with Diabetes and is provided by the University of Twente in association with the e-Vita program. The dataset consists of rows, each containing a user id, an action code, a timestamp and optionally extra information. The code describes what action the user performed, for example visiting the homepage or adding a personal goal. The main questions of this research are "What are the main predictors of adherence?" and "What are the predictors for the number of days between the first and second session?". To answer these questions, we first need to have a definition of adherence. Based on this definition, the adherence of the user will be determined which is used as the actual adherence value. To predict the adherence, a combination of attributes derived from the first session is adopted. A session is defined as all actions of a user that are within a 30-minute time span. Features are for example the usage of a certain function within e-Vita, the number of actions in the first session and the length of a session.

Furthermore, the predictors for the number of days between the first and second session are researched. This is based on the same attributes and uses regression to test the influence of the attributes.