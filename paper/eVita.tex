\documentclass{sig-alternate-br} % Use the layout as requested by the university

%===== Used packages =====
\usepackage[utf8]{inputenc}	% Use UTF8 characters
\usepackage{url}			% Proper url in the references section
\usepackage{flushend}		% Balance the columns on the last page
\usepackage{relsize}		% Provide \mathlarger to get formula's to the correct size
\usepackage{enumitem}		% Numbered items
\usepackage{placeins}		% Used for \FloatBarrier
\usepackage{float}
\usepackage{verbatim}		% Used for \begin{comment} \end{comment}
\setlist{itemsep=0pt, topsep=2pt}
\usepackage{hyperref}		% Internal references can be clicked
\pagenumbering{arabic} 	% Page numbering
\usepackage{booktabs} % better tables
% Setup captions styling
\usepackage{caption}
\captionsetup[lstlisting]{format=plain, singlelinecheck=false, margin=0pt, font={bf,footnotesize}, justification=centering}
\captionsetup[figure]{format=plain, singlelinecheck=false, margin=0pt, font={bf,footnotesize}, justification=centering, aboveskip=3pt}
\captionsetup[table]{format=plain, singlelinecheck=false, margin=0pt, font={bf,footnotesize}, justification=centering, aboveskip=3pt}

%===== Code snippet styling =====
\newcommand{\code}[1]{\texttt{\small \color{inline}#1}} % \code command for inline snippets
\usepackage{listings}		% Code snippets
\usepackage{color}			% Code highlighting colors
\definecolor{dkgreen}{rgb}{0,0.6,0}
\definecolor{inline}{rgb}{0,0,0.5}
\definecolor{gray}{gray}{0.2}
\definecolor{codeText}{rgb}{0.1,0.1,0.1}
\definecolor{mauve}{rgb}{0.58,0,0.82}
\lstset{frame=tb,
  framerule=0.2pt,
  language=Java,
  aboveskip=-2mm,
  belowskip=0mm,
  showstringspaces=false,
  columns=flexible,
  basicstyle={\small\ttfamily\color{codeText}},
  numbers=left,
  numbersep=4pt,
  numberstyle=\tiny\color{gray},
  keywordstyle=\color{blue},
  commentstyle=\color{dkgreen},
  stringstyle=\color{mauve},
  breaklines=true,
  breakatwhitespace=true,
  tabsize=4,
  morekeywords={data, day, dateGiven, daygiven, activity, startTime, endTime, studentsets, room, coursename, teacher}
}
\lstset{language=C, escapechar=$} % Set default language to C
% Set titles for listing references
%\renewcommand\lstlistingname{Algorithm}
%\renewcommand\lstlistlistingname{Algorithms}
%\def\lstlistingautorefname{Algorithm}
% Enable using math mode in code
\lstset{
  mathescape,         
  literate={->}{$\rightarrow$}{2}
           {ε}{$\varepsilon$}{1}
}

\begin{document}

\title{Data Science: e-Vita adherence prediction}

\numberofauthors{2}
\author{
\alignauthor Thijs Wiefferink\\
       \affaddr{University of Twente}\\
       \affaddr{P.O. Box 217, 7500AE Enschede}\\
       \affaddr{The Netherlands}\\
       \email{t.w.wiefferink@student.utwente.nl}
\alignauthor Patrick van Looy\\
       \affaddr{University of Twente}\\
       \affaddr{P.O. Box 217, 7500AE Enschede}\\
       \affaddr{The Netherlands}\\
       \email{p.vanlooy@student.utwente.nl}
}
\date{\today}


\maketitle

\begin{abstract}
Abstract
\end{abstract}

\keywords{Data Science, e-Vita, Adherence prediction, retention prediction}

\section{Introduction}
Introduction
\section{Related Work}
Related work
\section{Materials and Methods} \label{section:materialsAndMethods}
In this section the transformation of the logging data received from e-Vita to the \code{.arff} format of Weka is described. Next to this determining the features and defining adherence is described.
\subsection{Given data} \label{subsection:givenData}
The provided data is an Excel file with four columns and 9415 rows. The columns contain the \emph{ResearchNumber}, \emph{DateTimeTag}, \code{Code} and \code{ExtraInformation}. Each row represents one action taken by the user on the e-Vita website. The research number is a unique identifier per user, with which we are able to group the data per user. The code column denotes which action the user has taken, as described in Table~\ref{table:codes}. The extra information field has information about the section that the user has visited, the type of measurement that is taken, or an actual measurement value.

\begin{table}[]
	\centering
	\caption{Used codes in the data, and their count}
	\label{table:codes}
	\begin{tabular}{@{}lll@{}}
		\toprule
		\textbf{Code} & \textbf{Description}                                                                              & \textbf{Count} \\ \midrule
		10            & Homepage                                                                                          & 3314           \\
		21            & Opening lab values                                                                                & 391            \\
		22            & Clicking explanation of certain lab value                                                         & 1702           \\
		30            & Opening monitoring                                                                                & 184            \\
		31            & Opening graph of certain measurement                                                              & 231            \\
		33            & Opening graph previous measurements                                                               & 79             \\
		34            & Opening target values                                                                             & 69             \\
		35            & Adding new measurement                                                                            & 162            \\
		40            & Clicking button for extra information                                                             & 477            \\
		50            & Opening coaching                                                                                  & 448            \\
		51            & Clicking button “New wish”                                                                        & 53             \\
		52            & Adding new wish                                                                                   & 26             \\
		53            & Clicking button “Choose goals”                                                                    & 50             \\
		54            & Adding new goal                                                                                   & 35             \\
		55            & Change goal status                                                                                & 5              \\
		56            & Adding new action                                                                                 & 22             \\
		57            & Click button “Evaluate action”                                                                    & 12             \\
		58            & Adding new evaluation                                                                             & 9              \\
		59            & Click button “Ask for coaching”                                                                   & 12             \\
		60            & \begin{tabular}[c]{@{}l@{}}Click button “Yes, the response of the \\ coach is clear”\end{tabular} & 8              \\
		70            & Opening function for extra information                                                            & 401            \\
		71            & Actual opening the extra information                                                              & 235            \\
		90            & Opening function for education                                                                    & 630            \\
		91            & Actual opening education modules                                                                  & 859            \\ \bottomrule
	\end{tabular}
\end{table}

To process the given data it first has been saved as a tab separated file from Excel. Then a Java program reads this file and executes a couple of steps to get the data structured in a way that is easy to process. The list below shows the steps.

\begin{enumerate}
	\item Read the tab separated file.
	\item Go through the tab separated file line by line, create an Action object for each line. An Action contains the user id, date, code and extra information fields. The Action object is added to a map that maps user ids to a set with their actions, in Java this is represented by the following structure:\\ \code{Map< String, SortedSet<Action> >}. The Action object of a user are sorted by date to make the next step easier.
	\item Group the list of actions for a user into Session objects. A Session object holds the user id and a SortedSet of Actions. This involves going through the Actions of a user, and adding them to a Session while the time between them is 30 minutes or less. If there is more time in between then it starts a new session. This will output the following data structure:\\ \code{Map< String, SortedSet<Session> >}. The Session objects are sorted by date, so that the first Session of a user is at the start.
\end{enumerate}

The described process will deliver an in-memory representation of the data, this helps to detect adherence and determine features later. A couple of statistics about the data after the processing of this step can be found in Table~\ref{table:dataStats}. The exact code used for processing the files can be found on GitHub\footnote{https://github.com/NLthijs48/eVita}.

\begin{table}[]
	\centering
	\caption{Basic statistics of the provided data}
	\label{table:dataStats}
	\begin{tabular}{@{}ll@{}}
		\toprule
		Users                       & 301   \\
		Actions                     & 9414  \\
		Actions per user (average)  & 31.28 \\
		Sessions                    & 1187  \\
		Sessions per user (average) & 3.94  \\ \bottomrule
	\end{tabular}
\end{table}

Since the adherence and features will use the code field quite a lot, some exploration has been done for this column. In Table~\ref{table:codes} the number of times a code is used in an action is listed.

\subsection{Adherence definition}
Adherence in the context of the e-Vita website means that the user interacts with the website in a way that we think the user is actually using the website for something, instead of just trying it once. This is a property of the user that we are trying to predict based on the behaviour of the user in the first session. For this process we first need to determine which users are actually adhered and which are not. The definition of adherence is given below, if any of the given conditions is true, then we consider the user to be adherent.

\begin{itemize}
	\item The user visits the the application four times or more a year.
	\item The user added at least one personal, health-related goal (code 52).
	\item The user has followed at least one education module (code 91).
	\item The user added measurements (code 35) on four different days.
\end{itemize}

Since the provided data is only for a single year, the requirement of logging in at least four times a year is weakened interpreted as follows: The number of sessions of a user needs to be at least once each 3 months between his first and last login. This means that a user that logged in twice in the period of two months, meets the requirement. But a user that logged in twice in seven months, does not meet the requirement. 

\subsection{Features} \label{subsection:features}
To give input to the classifiers of the Weka tool we need to define features and determine the values for them per user. Because we want to predict the adherence based on the first session of the user the features will only use the data of the first session as input. Following there is a description of the implemented feature attributes.

\subsubsection{Code frequency feature}
The first feature count the number of times a certain code occurs during the first session. For example if you give it the code \code{35} in the constructor, then it will count the number of times the user has added a new measurement (listed in Table \ref{table:codes}) in the first session of using e-Vita. This feature has been generated for all codes listed in Table \ref{table:codes}. Since using certain features might predict that the user will be adherent, these feature are valuable.

\subsubsection{Actions in the first session}
This feature counts the number of actions the user has performed in his first session. The number of actions in the first session could say something about how interested the user is, and might therefore predict the adherence.

\subsubsection{Session length}
The session length feature counts how many minutes the user used e-Vita in his first session. For this the same applies as for the actions feature, it might indicate that the user is more interested in the application if he spends more time.

\subsubsection{Days between sessions}
Two more features have been added, the number of days between the first and second session, and the number of days between the second and third session. These features do not purely use data from the first session, but will independently be used to predict the adherence of a user instead of the other features.


\subsection{Creating a Weka file}
As input the Weka tool requires an \code{.arff} file. This is a simply file format where you first define the columns and then provide the data. Using the structured data from Section \ref{subsection:givenData} and the features from Section~\ref{subsection:features} this file for Weka has been generated.

\subsection{Using Weka}


\section{Results}
For the final results the actions described in Section \ref{subsection:usingWeka} have been executed.

\subsection{Predictors of adherence: First session}
The ranking of the attributes gave the ranking as seen in Table \ref{table:adherenceFirstAttributes}. As already expected the code 91 is a valuable attribute, this attribute is also used for the actual adherence value. Code 90 has to be used before you can use code 91 because of the layout of the website, so it is good to see that this code is just below. Then the action count and session length are the best attributes, which are also expected to have influence. The attributes that are not mentioned in the table have a score of 0, which means they give (almost) no information.

\begin{table}[]
	\centering
	\caption{First session adherence \\attributes ranking}
	\label{table:adherenceFirstAttributes}
	\begin{tabular}{@{}ll@{}}
		\toprule
		\textbf{Score} & \textbf{Attribute} \\ \midrule
		0.3535         & code91             \\
		0.102          & code90             \\
		0.0465         & actionCount        \\
		0.0395         & sessionLength      \\
		0.257          & code10             \\
		0.0176         & code52             \\
		0.0105         & code53             \\ \bottomrule
	\end{tabular}
\end{table}

Next the \code{ADTree} classifier has been ran, the decision tree can be found in Listing \ref{listing:adherenceFirstTree} and the confusion matrix can be found in Table \ref{table:adherenceFirstMatrix}. Once again code 91 is important, but now also the actions per hour is used often in the tree. This is probably because the actions per hour combines the action count and session length attributes, so therefore it might be more valuable than deciding on only one of them. 

\begin{lstlisting}[caption={Adherence first session, ADTree decision tree}, label=listing:adherenceFirstTree, float=htpb, language=xml]
: 0.03
|	(1)code91 < 0.5: 0.399
|	|	(2)actionsPerHour < 177.5: -0.088
|	|	|	(3)code51 < 0.5: 0.035
|	|	|	|	(8)actionsPerHour < 35.5: -0.475
|	|	|	|	(8)actionsPerHour >= 35.5: 0.061
|	|	|	|	|	(9)code10 < 4.5: -0.056
|	|	|	|	|	(9)code10 >= 4.5: 0.789
|	|	|	(3)code51 >= 0.5: -1.042
|	|	|	(6)actionCount < 14.5: 0.044
|	|	|	(6)actionCount >= 14.5: -0.438
|	|	|	|	(7)actionsPerHour < 54.5: 0.397
|	|	|	|	(7)actionsPerHour >= 54.5: -0.652
|	|	(2)actionsPerHour >= 177.5: 0.593
|	|	|	(4)sessionLength < 0.5: -0.669
|	|	|	(4)sessionLength >= 0.5: 0.324
|	|	|	|	(5)code30 < 0.5: 1.042
|	|	|	|	(5)code30 >= 0.5: -0.663
|	(1)code91 >= 0.5: -3.364
\end{lstlisting}

The confusion matrix shows that 88 instances have been correctly classified as adherent and 136 instances have been correctly classified as not adherent. 58 instances have been classified as not adherent, but should be adherent (19.27\%), and 19 instances have been classified as adherent while they should not be (6.31\%). This brings the performance of the classification to 74.42\%.

\begin{table}[]
	\centering
	\caption{Adherence first session, \\confusion matrix}
	\label{table:adherenceFirstMatrix}
	\begin{tabular}{@{}ll|l@{}}
		\toprule
		\textbf{A} & \textbf{B} & \textless-- classified as \\ \midrule
		88         & 58         & a = Adherent              \\
		19         & 136        & b = NotAdherent           \\ \bottomrule
	\end{tabular}
\end{table}


\subsection{Predictors of adherence: Days between sessions}
For predicting the adherence based on the number of days between the first three sessions the attribute ranking and decision tree has been produced again. The ranking shows that both attributes do not have much information gain, so this already tells that the performance is probably worse. The classification with \code{BFTree} produces the tree as shown in Listing \ref{listing:adherenceSessions}. The tree only uses the number of days between the first and second session.

\begin{lstlisting}[caption={Adherence days between sessions, BFTree decision tree}, label=listing:adherenceSessions, float=htpb, language=xml]
loginTimeFirstSecond < 201.5: Adherent(133.82/127.27)
loginTimeFirstSecond >= 201.5: NotAdherent(27.73/12.18)
\end{lstlisting}

The confusion matrix shows that 100 instance have been correctly classified as adherent and 114 instance have been correctly classified as not adherent. 46 instances have been classified as not adherent, but should be adherent (15.28\%), and 41 instances have been classified as adherent while they should not be (13.92\%). This classification has a performance of 71.10\%.

\begin{table}[]
	\centering
	\caption{Adherence days between sessions, \\confusion matrix}
	\label{table:adherenceSessionsMatrix}
	\begin{tabular}{@{}ll|l@{}}
		\toprule
		\textbf{A} & \textbf{B} & \textless-- classified as \\ \midrule
		100        & 46         & a = Adherent              \\
		41         & 114        & b = NotAdherent           \\ \bottomrule
	\end{tabular}
\end{table}


\subsection{Prediction of days between sessions}
First the results for the prediction of the days between the first and second session. Listing \ref{listing:sessionsFirstSecond} shows the formula generated by the \code{LinearRegression} algorithm and Table \ref{table:sessionsFirstSecond} shows the performance of the regression formula. The generated formula has a correlation of 0.0499, which is very close to zero, this means that the prediction of the number of days is quite bad.

\begin{lstlisting}[caption={Formula prediction of days between session one and two}, label=listing:sessionsFirstSecond, float=htpb, language=xml]
loginTimeFirstSecond =
	-7.0851 * code10 +
	24.6264 * code21 +
	597.643  * code30 +
	-620.1051 * code31 +
	754.1389 * code34 +
	-106.3367 * code35 +
	46.7325 * code50 +
	-93.1106 * code51 +
	-24.8427 * code90 +
	96.2365
\end{lstlisting}

\begin{table}[]
	\centering
	\caption{Performance prediction of days between session one and two}
	\label{table:sessionsFirstSecond}
	\begin{tabular}{@{}ll@{}}
		\toprule
		\textbf{Performance indicator}  & \textbf{Value} \\ \midrule
		Correlation coefficient         & 0.0499         \\
		Mean absolute error             & 90.4414        \\
		Root mean squared error         & 128.9717       \\
		Relative absolute error         & 106.3579\%     \\
		Root relative squared error     & 109.8663\%     \\
		Total number of instances       & 181            \\
		Ignored class unknown instances & 120            \\ \bottomrule
	\end{tabular}
\end{table}

Now the results of the prediction of the days between the second and third sessions. Listing \ref{listing:sessionsSecondThird} shows the formula generated by the \code{LinearRegression} algorithm and Table \ref{table:sessionsSecondThird} shows the performance of the regression formula. The generated formula has a correlation of -0.1348, which is very close to zero, this means that the prediction of the number of days is quite bad. Note that the correlation is below zero for this regression, and for the other one it was above zero.

\begin{lstlisting}[caption={Formula prediction of days between session one and two}, label=listing:sessionsSecondThird, float=htpb, language=xml]
loginTimeSecondThird =
	-46.866  * code30 +
	6.3303 * actionCount +
	-2.9964 * sessionLength +
	58.1361
\end{lstlisting}

\begin{table}[]
	\centering
	\caption{Performance prediction of days between session one and two}
	\label{table:sessionsSecondThird}
	\begin{tabular}{@{}ll@{}}
		\toprule
		\textbf{Performance indicator}  & \textbf{Value} \\ \midrule
		Correlation coefficient         & -0.1348        \\
		Mean absolute error             & 92.7638        \\
		Root mean squared error         & 124.7705       \\
		Relative absolute error         & 106.4739\%     \\
		Root relative squared error     & 108.7412\%     \\
		Total number of instances       & 128            \\
		Ignored class unknown instances & 173            \\ \bottomrule
	\end{tabular}
\end{table}







\section{Discussion}
To mention:
\begin{itemize}
	\item Small data set
\end{itemize}

\subsection{Assumptions}
Assumptions
\begin{itemize}
	\item ...
\end{itemize}

\subsection{Corrections}
Corrections
\begin{itemize}
	\item ...
\end{itemize}

\subsection{Inconsistencies}
\begin{itemize}
	\item ...
\end{itemize}
\section{Conclusion}
To conclude, the results have shown that with this dataset it is possible to come up with a predictor that is able of classifying adherent users correctly in 74\% of the cases. This predictor is based on the first session of a specific user. We have seen that by using linear regression on the predictors of the number of days between the first and second session, and the second and third session, it is not possible to obtain a significantly correct result. The correlation coefficient was very close to zero, meaning that a good correlation for the obtained formula could not be guaranteed.
\section{Appendices}
Appendices
\section{Acknowledgements}
We thank e-Vita for providing a data set for this research, and we thank the University of Twente for providing tools and methods supporting this project.

\bibliographystyle{ieeetr}
\bibliography{eVita}

\end{document}
